\documentclass[colorlinks,linkcolor=true]{article}
\usepackage{hyperref,verbatim}
\usepackage[usenames, dvipsnames]{color}
\usepackage{amsmath}
\usepackage{tgpagella} % text only
\usepackage{mathpazo} 
\usepackage[margin=0.6in]{geometry}
\addtolength{\topmargin}{0in}


\newcommand{\deadline}{Midnight April 20}
\newcommand{\extdeadline}{Midnight April 21}
\newcommand{\total}{15}

%opening
\title{Machine Learning\\Homework 8 : Neural Networks and Logistic Regression\\(due \deadline)}
\author{}
\date{}

\begin{document}
	
	\maketitle
	
	\noindent\fbox{
		\parbox{\textwidth}{
			Instructions\\
			\begin{enumerate}

				\item The deadline for full score is \deadline. You can get 50\% credit for late submission (\extdeadline).
				\item Total marks = \total
				\item You have to type the assignment using a word processing engine, create a pdf and upload on the form. Please note that only pdf files will be accepted.
				\item All code/Jupyter notebooks must be put up as secret gists and linked in the created pdf submission. Again, only secret gists. Not public ones.
				\item Any instances of cheating/plagiarism will not be tolerated at all. 
				\item Cite all the pertinent references in IEEE format.
				\item The least count of grading would be 0.5 marks. 
				
			\end{enumerate}
		}
	}


\begin{enumerate}



\item \begin{enumerate}
	
\item Logistic Regression
\begin{enumerate}
	\item Implement a function for binary logistic regression using gradient descent \textbf{[2 marks]}
	\item Show the usage of your implementation on the IRIS dataset. We will only be making use of sepal-length and petal-width as the two features. We have only two classes - Setosa and Not-Setosa.  \textbf{[1 marks]}
	\item Plot the decision boundary \textbf{[1 marks]}
	\item Compare your implementation against sklearn's Logistic Regression \textbf{[1 marks]}

\end{enumerate}

\item Neural Networks
\begin{enumerate}
	\item Implement a neural network class that can be instantiated with: 
	\begin{enumerate}
		\item an input data matrix X containing samples as rows and features as columns
		\item a list containing number of hidden units in each hidden layer
		\item a list containing activation function to be used in eacyh layer: sigmoid, softmax, ReLU, or identity (or linear)
		\item a cost function
	\end{enumerate} 
	As an example: let us say we have an input data matrix of shape 100X3, we use 2 layers and the number of hidden units is: [4, 2, 1] where the last number (1) indicates number of units in the output layer, the activations we use for the three layers are: ['ReLU','ReLU','Linear'] 
	 \textbf{[1 marks]}
	 \item For this class define a method forward propagation \textbf{[2 marks]}
	 \item For this class define a method backward propagation which provides the derivative of weights wrt cost function \textbf{[2 marks]}
	\item  Implement a gradient descent based method to update the weights \textbf{[1 mark]}
	
\end{enumerate}

\item Show the usage of your defined neural network on:
\begin{enumerate}
	\item MNIST dataset where you shuffle the data once, and then use first 50\% of the data for training, next 20\% for validation and last 30\% of the data for testing. You must return the confusion matrix and overall test accuracy. You may choose the number of layers and activations as per your choice \textbf{[2 marks]}
	\item Housing price dataset as you've been using thus far in earlier assignments where you shuffle the data once, and then use first 50\% of the data for training, next 20\% for validation and last 30\% of the data for testing. You must return the RMSE and MAE. You may choose the number of layers and activations as per your choice \textbf{[2 marks]}
\end{enumerate}
\end{enumerate}
\end{enumerate}



	





\end{document}
