\documentclass[]{article}
\usepackage{hyperref}
\usepackage[usenames, dvipsnames]{color}

%opening
\title{CSE 301: Operating Systems\\Homework 2 \\(due Noon Sept 17)}
\author{}
\date{}

\begin{document}

\maketitle

\noindent\fbox{
	\parbox{\textwidth}{
		Instructions\\
		\begin{enumerate}
			\item The deadline is a hard one. The form upload will close at sharp noon Sept 17. There will be no extensions.
			\item You have to type the assignment using a word processing engine, create a pdf and upload on the form. Please note that only pdf files will be accepted.
			\item Name the submission as \texttt\{branch\}\_\{roll\_number\}\_\{name\}.pdf
			\item All code/Jupyter notebooks must be put up as \href{https://gist.github.com/}{\textbf{secret gists}} and linked in the created pdf. Again, only secret gists. Not the public ones.
			\item Any instances of cheating/plagiarism will not be tolerated at all. 
			\item Cite all the pertinent references in IEEE format.
			\item The least count of grading would be 0.5 marks. 
			\item Some suggestions for plotting - WolframAlpha, Academo.Org, Geogebra, Matploltib, GNUplot, Matlab, Octave
			\item You can find the course VM on the course web page. The root password is: 1234
		\end{enumerate}
	}
}


\begin{enumerate}
	\item Create a Jupyter notebook to compare the efficacy of FIFO, Random, LRU and Optimal (Oracle) page replacement algorithms on the 90-10 workload: 90\% of the references are to 10\% of the pages and the remaining 10\% of the references are to the remaining 90\% of the pages. Make a plot showing hit rate v/s cache size for the different algorithms. \textbf{[3 marks]}
	
	Some hints/suggestions/assumptions:
	\begin{enumerate}
		\item Keep the workload size as 1000 (number of page accesses)
		\item Assume a total of 100 unqiue page numbers
		\item Now, you can create a very large number of workloads satisfying the constraint of 90-10 load type, workload size, and number of unique pages. To keep things simple, just use any such possible workload. 
		\item You can use the
		 \underline{\href{https://github.com/nipunbatra/nipunbatra.github.io/blob/master/teaching/os-fall-18/code/replacement-random-plot.py}{instructor provided sample code for Random}} to get started
	\item For random, unlike other algorithms, your hit rate would depend on the random seed. Run the random page replacement for different random seeds and plot not only the mean but also the standard deviation of the hit rate
	\item Show a proof of the workload being 90-10. For this, write a simple code snippet that plots the count of each page in the workload. Then, create another plot that counts the cumulative count of the "hot" pages and that of "non-hot" pages.
	\item You can vary the cache size as : [1, 2, 3, 5, 10, 15, 20, 25, 30, 35, 40...1000]
	\end{enumerate}



	
	
	\item Consider a system with a 6 bit virtual address space, and 16 byte pages/frames. The mapping
	from virtual page numbers to physical frame numbers of a process is (0,9), (1,1), (2,6), and
	(3,8). Translate the following virtual addresses to physical addresses. Note that all addresses
	are in decimal. You may write your answer in decimal or binary.  
	(a) 22
	(b) 62
	(c) 109 \textbf{[1.5 marks]}
	
	\item Question 3 and 4 from OSTEP Chapter 13: Create a little program that uses a certain amount of memory, called
	memory-user.c. This program should take one command-line argument:
	the number of megabytes of memory it will use. When run, it should allocate
	an array, and constantly stream through the array, touching each entry.
	The program should do this indefinitely, or, perhaps, for a certain amount
	of time also specified at the command line.
	Now, while running your memory-user program, also (in a different terminal
	window, but on the same machine) run the free tool. How do the
	memory usage totals change when your program is running? How about
	when you kill the memory-user program? Do the numbers match your expectations?
	Try this for different amounts of memory usage. What happens
	when you use really large amounts of memory?  \textbf{[1.5 marks]}
	
\end{enumerate}


Not for grading
\begin{enumerate}
\item Write a program that takes following inputs: a) workload size (n), b) num unique pages (np), c) cache size 1 (c1) and returns the workload sequence which shows Belady's anomaly for c1 and c1 + 1. For example, given n = 12, np = 5, c1 = 3, we have: the sequence 1, 2, 3, 4, 1, 2, 5, 1, 2, 3, 4, 5 for which we have more hits for cache size, c1 = 3, compared to c1 + 1 =4.
\end{enumerate}









\end{document}
