\documentclass[]{article}
\usepackage{hyperref}

%opening
\title{OS course Fall 2018\\Lab 1}
\date{}

\begin{document}

\maketitle
\section{Unix commands}
\begin{enumerate}
	\item Fire the terminal and run the \texttt{ps} command on the terminal without any arguments. What output do you see? Where is this program being run?
	\item Fire up a program. Can you see it using \texttt{ps}?
	\item Look at the manual for \texttt{ps} using \texttt{man ps}. Try out the command with all the flags and note the output in each case.
	\item How can you get details of \textbf{all} the running processes?
	\item Which order are the above processes sorted by? Can you sort them using same criteria but opposite order?
	\item Can you sort them using some other criteria? i.e. by CPU usage?
	\item Use \texttt{ps} to find unique users?
	\item Find number of process per user.
	\item Find process by PID
	\item Find its parent
	\item Find details of multiple processes using PID list
	\item Use \texttt{-o} to output PID, PPID
	\item Find total running time of all processes
	\item Sort processes in decreasing order of running time
	\item Use \texttt{watch} utility to perform real-time monitoring over ps\\
	\texttt{watch -n 1 `ps -eo pid,ppid,cmd,\%mem,\%cpu --sort=-\%mem | head'}
	\item Change the monitoring period of \texttt{watch}
\end{enumerate}

\section{Process API}
In this part of the lab, we would be running the 8 programs in the homework section of \href{http://pages.cs.wisc.edu/~remzi/OSTEP/cpu-api.pdf}{Chapter 5} from the textbook.





\end{document}
